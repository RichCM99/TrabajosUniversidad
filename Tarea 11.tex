\documentclass[10pt,a4paper]{article}
\usepackage[latin1]{inputenc}
\usepackage[spanish]{babel}

% Para mejores m�rgenes
\usepackage{geometry}
\usepackage{fancyhdr} %este lo puse yo
\usepackage{setspace} %este tambien

% Paquetes est�ndar de matem�ticas
\usepackage{amsmath,amssymb,amsfonts}
\usepackage{mathrsfs} % para la familia \mathscr
\usepackage{derivative} % lo puse yo

% Teoremas 
\usepackage{amsthm}

\theoremstyle{plain}
\newtheorem{thm}{Teorema}
\newtheorem{prop}{Proposici�n}
\newtheorem{lem}{Lema}

\theoremstyle{definition}
\newtheorem{dfn}{Definici�n}
\newtheorem{pre}{Pregunta}

\theoremstyle{remark}
\newtheorem{obs}{Observaci�n}

% Respuesta
\newenvironment{prueba}{\renewcommand{\proofname}{Prueba}\renewcommand{\qedsymbol}{}\begin{proof}}{\end{proof}}

% Enumeraci�n
\usepackage{enumitem}

% Tablas bonitas
\usepackage{booktabs}

% Diagramas
\usepackage{tikz-cd}

% Verbatim
\usepackage{fancyvrb}

% Color y Links
\usepackage{xcolor}
\definecolor{mine}{RGB}{0,0,180}
\usepackage{hyperref}
\hypersetup{final,hidelinks, colorlinks, linkcolor = mine, citecolor = mine,
  urlcolor = mine}


%%% Macros

%% Valor absoluto y norma
\newcommand{\abs}[1]{\left\lvert #1 \right\rvert}
\newcommand{\norm}[1]{\left\lVert #1 \right\rVert}

% Conjuntos de n�meros
\newcommand{\R}{\mathbb R}
\newcommand{\Q}{\mathbb Q}
\newcommand{\Z}{\mathbb Z}
\newcommand{\N}{\mathbb N}
\newcommand{\C}{\mathbb C}

\def\r{\mathbb{R}}
\def\rn{\mathbb{R}^n}
\def\z{\mathbb{Z}}
\def\n{\mathbb{N}}
\def\s{\mathbb{S}}
\def\eps{\varepsilon}
\def\o{\Omega}
\def\vp{\varphi}
\def\rh{\rightharpoonup}
\def\io{\int_{\Omega}}
\def\irn{\int_{\r^N}}
\def\d{\mathrm{d}}
\def\wt{\widetilde}
\def\wh{\widehat}
\def\cB{\mathcal{B}}
\def\cC{\mathcal{C}}
\def\cD{\mathcal{D}}
\def\cH{\mathcal{H}}
\def\cI{\mathcal{I}}
\def\cJ{\mathcal{J}}
\def\cK{\mathcal{K}}
\def\cL{\mathcal{L}}
\def\cM{\mathcal{M}}
\def\cN{\mathcal{N}}
\def\cO{\mathcal{O}}
\def\cP{\mathcal{P}}
\def\cT{\mathcal{T}}
\def\cU{\mathcal{U}}
\def\cV{\mathcal{V}}
\def\cW{\mathcal{W}}
\def\sin{\mathrm{sen}}
\def\id{\mathrm{id}}

\title{Tarea 11}
\date{\today}
\author{Ricardo Cruz Mart�nez}

\begin{document}
\maketitle

\begin{enumerate}

\item Sea $f\in\cC^0[0,1]$ tal que
  \begin{equation*}
    \int_0^1f(x)\,x^n\,\d x=0\qquad\forall n\in\n\cup\{0\}.
  \end{equation*}
Prueba que $f(x)=0$ para todo $x\in [0,1]$.
\begin{proof}
Por el Teorema de Aproximaci�n de Weierstrass, $\exists (p_{k})$ una sucesi�n de polinomios en $[0,1]$ que converge uniformemente a $f$.\\
Por otra parte, sea $p:[0,1]\longrightarrow{\R}$ un polinomio.\\
 Supongamos que $p(x)=a_{0}+a_{1}x+\cdots+a_{n}x^{n}=\displaystyle\sum_{i=0}^{n}a_{i}x^{i}$, p.a. $a_{1},\cdots,a_{n}\in{\R}$\\
 Notemos que
 \begin{center}
 $f(x)p(x)=\displaystyle\sum_{i=0}^{n}f(x)a_{i}x^{i}$
 \end{center}
 Seguido de esto, por linealidad de la integral y por hip�tesis tenemos que 
 \begin{center}
 $\displaystyle\int_{0}^{1}f(x)p(x)dx=\int_{0}^{1}\sum_{i=0}^{n}f(x)a_{i}x^{i}dx=\sum_{i=0}^{n}a_{i}\int_{0}^{1}f(x)x^{i}dx=\sum_{i=0}^{n}a_{i}\cdot 0=0$
 \end{center}
 De esto �ltimo, concluimos que $\displaystyle\int_{0}^{1}f(x)p(x)dx=0$ para todo polinomio definido en $[0,1]$.\\
 As�
 \begin{center}
 $\displaystyle\int_{0}^{1}f(x)p_{k}(x)dx=0\quad\forall k\in{\N}$
 \end{center}
 Y como $p_{k}\longrightarrow f$ uniformemente en $\cC^{0}([0,1])$, se sigue que
 \begin{center}
 $0=\displaystyle\lim_{k\longrightarrow\infty}\int_{0}^{1}f(x)p_{k}(x)dx=\int_{0}^{1}\lim_{k\longrightarrow\infty}f(x)p_{k}(x)dx=\int_{0}^{1}f(x)\lim_{k\longrightarrow\infty}p_{k}(x)dx=\int_{0}^{1}f^{2}(x)dx$
 \end{center}
 $\therefore\displaystyle\int_{0}^{1}f^{2}(x)dx=0$.\\
 Adem�s, $f^{2}(x)\geq 0\quad\forall x\in[0,1]$, y como $f$ es continua por hip�tesis, entonces $f^{2}$ tambi�n lo es, por lo que concluimos que $f(x)=0\quad\forall x\in[0,1]$
\end{proof}

\item Sea $\s^1=\{(\cos\theta,\sen\theta)\in\r^2:0\leq\theta<2\pi\}$ el c�rculo unitario en $\r^2$. Prueba que cualquier funci�n continua $f:\s^1\to\r$ es el l�mite uniforme de funciones de la forma
  \begin{equation*}
    \varphi(\cos\theta,\sen\theta)=a_0+a_1\cos\theta+b_1\sen\theta+\cdots+a_n\cos n\theta+b_n\sen n\theta
  \end{equation*}
con $a_i,b_i\in\r$, $n\in\n\cup\{0\}$.
\begin{spacing}{1.4}
\begin{proof}
Notemos que $\s^{1}$ es compacto, pues $\s^{1}=fr(B_{1}(0))$, la cual siempre es cerrada (la frontera de un conjunto siempre es cerrada) en ${\R}^{2}$ y claramente es acotada en ${\R}^{2}$, as�, por el Teorema de Heine-Borel, tenemos que $\s^{1}$ es compacto en ${\R}^{2}$.\\
Definimos 
$A=\{\phi\in\cC^{0}(\s^{1}):\phi(\cos(\theta),sen(\theta))=a_0+a_1\cos\theta+b_1\sen\theta+\cdots+a_n\cos n\theta+b_n\sen n\theta,\quad n\in{\N}\cup\{0\},\theta\in[0,2\pi),a_{i},b_{i}\in{\R}\}$\\
Donde $A\subset\cC^{0}(\s^{1})$\\
Veamos que $A$ cumple las hip�tesis del Teorema de Stone Weierstrass
\begin{enumerate}
\item Sean $\varphi,\eta\in A$, $\alpha,\lambda\in{\R}$\\
Queremos ver que $\alpha\varphi+\lambda\eta\in A$\\
Como $\varphi\in A$, $\exists n\in{\N}\cup\{0\}$, $\exists a_{i},b_{i}\in{\R}$, $i\in\{1,\cdots,n\}$ tales que 
\begin{center}
$\varphi(\cos(\theta),\sen(\theta))=a_{0}+a_{1}\cos(\theta)+\cdots+a_{n}\cos(n\theta)+b_{n}\sen(n\theta)$
\end{center}
Y como $\eta\in A$, $\exists k\in{\N}\cup\{0\}$, $\exists c_{i},d_{i}\in{\R}$, $i\in\{1,\cdots k\}$ tales que 
\begin{center}
$\eta(\cos(\theta),\sen(\theta))=c_{0}+c_{1}\cos(\theta)+\cdots+c_{k}\cos(k\theta)+d_{k}\sen(k\theta)$
\end{center}
Adem�s, tenemos que 
\begin{center}
$\alpha\varphi(\cos(\theta),\sen(\theta))=\alpha a_{0}+\alpha a_{1}\cos(\theta)+\cdots+\alpha a_{n}\cos(n\theta)+\alpha b_{n}\sen(n\theta)$\\
$\lambda\eta(\cos(\theta),\sen(\theta))=\lambda c_{0}+\lambda c_{1}\cos(\theta)+\cdots+\lambda c_{k}\cos(k\theta)+\lambda d_{k}\sen(k\theta)$
\end{center}
Sin perder generalidad, supongamos que $k\leq n$, por lo que de lo anterior tenemos lo siguiente
\begin{equation*}
\begin{split}
(\alpha\varphi+\lambda\eta)(\cos(\theta),\sen(\theta)) & = (\alpha a_{0}+\lambda c_{0})+(\alpha a_{1}+\lambda c_{1})\cos(\theta)+\cdots+(\alpha a_{k}+\lambda c_{k})\cos(k\theta)\\ 
& + (\alpha b_{k}+\lambda d_{k})\sen(k\theta)+\alpha a_{k+1}\cos((k+1)\theta)+\alpha b_{k+1}\sen((k+1)\theta)+\cdots\\
& + \alpha a_{n}\cos(n\theta)+\alpha b_{n}\sen(n\theta)
\end{split}
\end{equation*}
Y notemos que $\alpha\varphi+\lambda\eta\in A$

\item Sean $\varphi,\eta\in A$\\
Queremos ver que $\varphi\eta\in A$\\
Como $\varphi\in A$, $\exists n\in{\N}\cup\{0\}$, $\exists a_{i},b_{i}\in{\R}$, $i\in\{1,\cdots,n\}$ tales que 
\begin{center}
$\varphi(\cos(\theta),\sen(\theta))=a_{0}+a_{1}\cos(\theta)+\cdots+a_{n}\cos(n\theta)+b_{n}\sen(n\theta)$
\end{center}
Y como $\eta\in A$, $\exists k\in{\N}\cup\{0\}$, $\exists c_{i},d_{i}\in{\R}$, $i\in\{1,\cdots k\}$ tales que 
\begin{center}
$\eta(\cos(\theta),\sen(\theta))=c_{0}+c_{1}\cos(\theta)+\cdots+c_{k}\cos(k\theta)+d_{k}\sen(k\theta)$
\end{center}
Nuevamente, sin perder generalidad, supongamos que $n\leq k$\\
De Variable Compleja 1, sabemos que las funciones $\sen,\cos$ definidas en los complejos, son una generalizaci�n de estas mismas funciones pero en los reales, donde 
\begin{center}
$\sen(n\theta)=\dfrac{e^{in\theta}-e^{-in\theta}}{2i}$ \quad y $\cos(n\theta)=\dfrac{e^{in\theta}+e^{-in\theta}}{2}$ 
\end{center} 
Veamos que ocurre cuando multiplicamos $\sen(j\theta)\cos(k\theta)$ con $j,k\in\{1,\cdots n\}$
\begin{equation*}
\begin{split}
\sen(j\theta)\cos(k\theta) & = \dfrac{(e^{ij\theta}-e^{-ij\theta})(e^{ik\theta}+e^{-ik\theta})}{4i} \\
& = \dfrac{e^{i(j+k)\theta}-e^{i(-j+k)\theta}+e^{i(j-k)\theta}-e^{-i(j+k)\theta}}{4i}\\
& = \dfrac{e^{i(j+k)\theta}-e^{-i(j+k)\theta}}{4i}+\dfrac{e^{i(j-k)\theta}-e^{-i(j-k)\theta}}{4i}\\
& = \frac{1}{2}\sen((j+k)\theta)+\frac{1}{2}\sen((j-k)\theta)
\end{split}
\end{equation*}
Ahora, veamos que da cuando multiplicamos $\sen(j\theta)\sen(k\theta)$ con $j,k\in\{1,\cdots n\}$ 
\begin{equation*}
\begin{split}
\sen(j\theta)\sen(k\theta) & = \dfrac{(e^{ij\theta}-e^{-ij\theta})(e^{ik\theta}-e^{-ik\theta})}{(2i)(2i)} \\
& = \dfrac{-e^{i(j+k)\theta}+e^{i(-j+k)\theta}+e^{i(j-k)\theta}-e^{-i(j+k)\theta}}{4}\\
& = -\dfrac{e^{i(j+k)\theta}+e^{-i(j+k)\theta}}{4}+\dfrac{e^{i(-j+k)\theta}+e^{-i(-j+k)\theta}}{4}\\
& = -\frac{1}{2}\cos((j+k)\theta)+\frac{1}{2}\cos((-j+k)\theta)
\end{split}
\end{equation*} 
Y finalmente veamos que da cuando multiplicamos $\cos(j\theta)\cos(k\theta)$ con $j,k\in\{1,\cdots n\}$
\begin{equation*}
\begin{split}
\cos(j\theta)\cos(k\theta) & = \dfrac{(e^{ij\theta}+e^{-ij\theta})(e^{ik\theta}+e^{-ik\theta})}{4} \\
& = \dfrac{e^{i(j+k)\theta}+e^{i(-j+k)\theta}+e^{i(j-k)\theta}+e^{-i(j+k)\theta}}{4}\\
& = \dfrac{e^{i(j+k)\theta}+e^{-i(j+k)\theta}}{4}+\dfrac{e^{i(-j+k)\theta}+e^{-i(-j+k)\theta}}{4}\\
& = \frac{1}{2}\cos((j+k)\theta)+\frac{1}{2}\cos((-j+k)\theta)
\end{split}
\end{equation*}
Por ende, cuando hacemos el producto de $\varphi\eta$, como cada t�rmino de la multiplicaci�n es de la forma $a_{i}c_{j}\cos(i\theta)\cos(j\theta)$, $a_{i}d_{j}\cos(i\theta)\sen(j\theta)$, $b_{i}c_{j}\sen(i\theta)\cos(j\theta)$ � $b_{i}d_{j}\sen(i\theta)\sen(j\theta)$ con $i\in\{1,\cdots,n\},j\in\{1,\cdots,m\}$ y donde ya vimos que dichos productos se quedan en $A$, entonces la suma y multiplicaci�n por escalares tambi�n.\\
$\therefore \varphi\eta\in A$.

\item Claramente $\varphi(\cos(\theta),\sen(\theta))=1\in A$

\item Sean $x_{1},x_{2}\in\s^{1}$, $x_{1}\neq x_{2}$\\
 Queremos ver que $\exists\zeta\in A$ tal que $\zeta(x_{1})\neq\zeta(x_{2})$\\
Primero, consideremos 
\begin{center}
$\beta(\cos(\theta),\sen(\theta))=\cos(\theta)+\sen(\theta)$ y\\
$\gamma(\cos(\theta),\sen(\theta))=\cos(\theta)-\sen(\theta)$
\end{center}
Donde claramente $\beta,\gamma\in A$, veamos que para $x_{1}$ y $x_{2}$ la imagen de estos puntos es distinta bajo alguna de estas dos funciones.\\
Como $x_{1}\in\s^{1}\Rightarrow\exists\theta_{1}\in[0,2\pi)$ tal que $x_{1}=(\cos(\theta_{1}),\sen(\theta_{2}))$, an�logamente como $x_{2}\in\s^{1}\Rightarrow\exists\theta_{2}\in[0,2\pi)$ tal que $x_{2}=(\cos(\theta_{2}),\sen(\theta_{2}))$.\\
Por otra parte, dados $\theta,\varphi\in[0,2\pi)$, si $\beta(\cos(\theta),\sen(\theta))=\beta(\cos(\varphi),\sen(\varphi))$ y adem�s $\gamma(\cos(\theta),\sen(\theta))=\gamma(\cos(\varphi),\sen(\varphi))$, tenemos lo siguiente
\begin{gather}
\cos(\theta)+\sen(\theta)=\cos(\varphi)+\sen(\varphi)\\
\cos(\theta)-\sen(\theta)=\cos(\varphi)-\sen(\varphi)
\end{gather}
De $(2)$, tenemos que 
\begin{center}
$\cos(\theta)=\cos(\varphi)-\sen(\varphi)+\sen(\theta)$
\end{center} 
Y sustituyendo en $(1)$ se sigue que
\begin{center}
$\cos(\varphi)-\sen(\varphi)+\sen(\theta)+\sen(\theta)=\cos(\varphi)+\sen(\varphi)$\\
$\Rightarrow 2\sen(\theta)=2\sen(\varphi)\Rightarrow\sen(\theta)=\sen(\varphi)$
\end{center}
Ahora sustituyendo esto en $(2)$
\begin{center}
$\cos(\theta)-\sen(\varphi)=\cos(\varphi)-\sen(\varphi)$\\
$\Rightarrow \cos(\theta)=\cos(\varphi)$
\end{center}
Ahora, como $x_{1}\neq x_{2}$, tenemos que $(\cos(\theta_{1}),\sen(\theta_{1}))\neq(\cos(\theta_{2}),\sen(\theta_{2}))$, de esto se sigue que $\cos(\theta_{1})\neq\cos(\theta_{2})$ � $\sen(\theta_{1})\neq\sen(\theta_{2})$, con esto �ltimo, concluimos que $x_{1}$ y $x_{2}$ no pueden dar al mismo punto bajo $\beta$ y $\gamma$, por lo que dependiendo el caso, $\zeta=\beta$ � $\zeta=\gamma$, por lo que $\zeta(x_{1})\neq\zeta(x_{2})$ y $\zeta\in A$ 
\end{enumerate} 
Por ende, de los incisos $a),b),c)$ y $d)$, concluimos que $A$ es denso en $\cC^{0}(\s^{1})$ (ya que tenemos las hip�tesis del Teorema de Stone-Weierstrass), as�, dada una funci�n continua $f:\s^{1}\longrightarrow {\R}$, existe una sucesi�n $(\varphi_{k})$ de funciones en $A$ que converge uniformemente a $f$ en $\s^{1}$
\end{proof}
\end{spacing}

\item Sean $V,W,Z$ espacios de Banach. Demuestra las siguientes dos afirmaciones.
  \begin{enumerate}
  \item[$(a)$] Si $S\in\cL(V,W)$ y $T\in\cL(W,Z)$ entonces
    \begin{equation*}
      \|T\circ S\|_{\cL(V,Z)}\leq \|T\|_{\cL(W,Z)}\|S\|_{\cL(V,W)}.
    \end{equation*}
    \begin{spacing}{1.7}
    \begin{proof}
    Sean $S\in\cL(V,W)$, $T\in\cL(W,Z)$.\\
    Claramente tenemos que se cumplen las siguientes desigualdades
    \begin{center}
    $\norm{T(a)}_{Z}\leq\norm{T}_{\cL(W,Z)}\norm{a}_{W}\quad\forall a\in W\quad\forall T\in\cL(W,Z)\cdots(1)$  \\
    $\norm{S(b)}_{W}\leq\norm{S}_{\cL(V,W)}\norm{b}_{V}\quad\forall b\in V\quad\forall S\in\cL(V,W)\cdots(2)$
    \end{center}
    Sea $v\in V$\\
    Por las desigualdades $(1)$ y $(2)$, se tiene que 
    \begin{center}
    $\norm{T(S(v))}_{Z}\leq\norm{T}_{\cL(W,Z)}\norm{S(v)}_{W}\leq\norm{T}_{\cL(W,Z)}\norm{S}_{\cL(V,W)}\norm{v}_{V}$
    \end{center}
    $\therefore\norm{T(S(v))}_{Z}\leq\norm{T}_{\cL(W,Z)}\norm{S}_{\cL(V,W)}\norm{v}_{V}\quad\forall v\in V$.\\
    Adem�s, si $v\neq 0_{V}$, se tiene que 
    \begin{center}
    $\dfrac{\norm{T\circ S(v)}_{Z}}{\norm{v}_{V}}\leq\norm{T}_{\cL(W,Z)}\norm{S}_{\cL(V,W)}\quad\forall v\in V, v\neq 0_{V}$
    \end{center}
    De esto �ltimo, tenemos que 
    \begin{center}
    $\displaystyle\norm{T\circ S}_{\cL(V,Z)}=\sup_{v\in V, v\neq 0}\dfrac{\norm{T\circ S(v)}_{Z}}{\norm{v}_{V}}\leq\norm{T}_{\cL(W,Z)}\norm{S}_{\cL(V,W)}$
    \end{center}
    \end{proof}
    \end{spacing}
  \item[$(b)$] Si $S_k\to S$ en $\cL(V,W)$ y $T_k\to T$ en $\cL(W,Z)$, entonces $T_k\circ S_k\to T\circ S$ en $\cL(V,Z)$.
  \begin{spacing}{1.5}
  \begin{proof}
  Sea $\varepsilon>0$\\
  Como $S_{k}\longrightarrow S$, entonces $\exists N_{1}\in{\N}$ tal que $\forall k\in{\N}$, si $k\geq N_{1}\Rightarrow\norm{S_{k}-S}_{\cL(V,W)}<\varepsilon_{1}$.\\
  Como $T_{j}\longrightarrow T$, entonces $\exists N_{2}\in{\N}$ tal que $\forall j\in{\N}$, si $j\geq N_{2}\Rightarrow\norm{T_{j}-T}_{\cL(W,Z)}<\varepsilon_{2}$.\\
  Donde $\varepsilon_{1}=\dfrac{\varepsilon}{2\norm{T}_{\cL(W,Z)}}$ y $\varepsilon_{2}=\dfrac{\varepsilon}{2\norm{S_{n}}_{\cL(V,W)}}$.\\
  Notemos que, como $S_{n}\longrightarrow S$, entonces $S_{n}$ est� acotada y por ende $\norm{S_{n}}_{\cL(V,W)}<\infty$, por lo que tiene sentido como definimos $\varepsilon_{2}$.\\
  Y de las desigualdades del inicio, se tiene que 
  \begin{center}
  $\norm{S_{k}(v)-S(v)}_{W}<\varepsilon_{1}\norm{v}_{V}\quad\forall k\geq N_{1}\quad\forall v\in V\cdots(1)$\\
  $\norm{T_{j}(w)-T(w)}_{Z}<\varepsilon_{2}\norm{w}_{W}\quad\forall j\geq N_{2}\quad\forall w\in W\cdots(2)$
  \end{center}
  Sean $N=\max\{N_{1},N_{2}\}$, $n\in{\N}$ tal que $n\geq N$ y sea $v\in V$\\
  Queremos ver que $\norm{T_{n}\circ S_{n}(v)-T\circ S(v)}_{Z}<\varepsilon\norm{v}_{V}$.\\
  $\norm{T_{n}\circ S_{n}(v)-T\circ S(v)}_{Z}\leq\norm{T_{n}\circ S_{n}(v)-T\circ S_{n}(v)}_{Z}+\norm{T\circ S_{n}(v)-T\circ S(v)}_{Z}$.\\
  Como $S_{n}\in{\N}\quad\forall n\in{\N}$ y adem�s $n\geq N\geq N_{2}$, tenemos que 
  \begin{center}
  $\norm{T_{n}\circ S_{n}(v)-T\circ S_{n}(v)}_{Z}<\varepsilon_{2}\norm{S_{n}(v)}_{W}\leq\varepsilon_{2}\norm{S_{n}}_{\cL(V,W)}\norm{v}_{V}$
  \end{center}   
  La �ltima desigualdad se desprende de lo mencionado al inicio del inciso $a)$.\\
  Por otra parte, como $n\geq N\geq N_{1}$, tenemos que
  \begin{center}
  $\norm{T\circ S_{n}(v)-T\circ S(v)}_{Z}\leq\norm{T}_{\cL(W,Z)}\norm{S_{n}(v)-S(v)}_{W}<\varepsilon_{1}\norm{T}_{\cL(W,Z)}\norm{v}_{V}$
  \end{center}
  Por lo que finalmente tenemos que $\forall n\geq N$
  %\end{spacing}
  \begin{equation*}
  \begin{split}
  \norm{T_{n}\circ S_{n}(v)-T\circ S(v)}_{Z} & <\varepsilon_{2}\norm{S_{n}}_{\cL(V,W)}\norm{v}_{V}+\varepsilon_{1}\norm{T}_{\cL(W,Z)}\norm{v}_{V} \\
  & = \dfrac{\varepsilon\norm{v}_{V}}{2}+\dfrac{\varepsilon\norm{v}_{V}}{2}=\norm{v}_{V}\varepsilon
  \end{split}
  \end{equation*}
As� $\norm{T_{n}\circ S_{n}(v)-T\circ S(v)}_{Z}<\varepsilon\norm{v}_{V}\quad\forall v\in V$\\
$\therefore T_{n}\circ S_{n}\longrightarrow T\circ S$    
  \end{proof}
  \end{spacing}
    \end{enumerate}

\end{enumerate}
  
\end{document}